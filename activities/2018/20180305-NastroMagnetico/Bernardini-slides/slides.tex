%
% $Id: slides.tex 49 2014-02-16 10:42:33Z nicb $
%
% Copyright (C) 2006 Nicola Bernardini nicb@sme-ccppd.org
% 
% This work is licensed under a Creative Commons License, and specifically the
% 
%   Creative Commons Attribution-ShareAlike 2.5 License
%   http://creativecommons.org/licenses/by-sa/2.5/legalcode
% 
% Check http://www.creativecommons.org/ for more information on
% Creative Commons Licenses and the Creative Commons Project.
%
% Set the macros below to whatever is appropriate in a given context
%

\newcommand{\scelsiroot}{/home/nicb/me/svn/nicb/projects/music/Scelsi/doc/presentazioni}
\newcommand{\imagedir}{\scelsiroot/materiali}
\newcommand{\exampledir}{\scelsiroot/materiali}
\documentclass[compress,xcolor=dvipsnames]{beamer}

\usepackage{beamerthemeSME-CCPPD}
\usepackage{beamercolorthemeSME-CCPPD}
\usepackage{beamerinnerthemeSME-CCPPD}

\usepackage{colortbl}
\usepackage[italian]{babel}
\usepackage{pgf}
\usepackage{xspace}

\usepackage{multimedia}
\usepackage{xmpmulti}
\usepackage{hyperref}
\usepackage{gitinfo2}
\newcommand{\rcstag}{v.\gitAbbrevHash}
\usepackage{gensymb}
\usepackage[12pt]{moresize}

\newcommand{\hhref}[1]{\href{#1}{#1}}

\newcommand{\cpyear}{2018}
\newcommand{\cpholder}{Nicola Bernardini}
\newcommand{\cpholderemail}{nicb@sme-ccppd.org}

% Use some nice templates

%\beamertemplateshadingbackground{red!10}{structure!10}
\beamertemplatetransparentcovereddynamic
\beamertemplateballitem
\beamertemplatenumberedballsectiontoc

% My colors
\definecolor{notdone}{gray}{0.35}

%\usecolortheme[named=MyColor]{structure}
%\usecolortheme[named=MyColor]{structure}
\beamertemplateshadingbackground{white!10}{white!10}

\setbeamerfont{smalllist}{size=\ssmall}

\newcommand{\FIS}{Fondazione Isabella Scelsi\xspace}
\newcommand{\Mo}{M\degree\xspace}
\newcommand{\Scelsi}{Giacinto Scelsi\xspace}
 
\title[Recovering Scelsi's Tapes]%
{%
  I nastri di Giacinto Scelsi\\
	{\tiny (\rcstag)}
}

\author{%
	Nicola Bernardini\\
    \href{mailto:\cpholderemail}{\cpholderemail}
}
\institute[SME-CCPPD]%
{%
	\href{http://www.scelsi.it}{Fondazione Isabella Scelsi -- Roma}\\
	\href{http://www.conservatoriopollini.it}
		 {Conservatorio di Musica ``C.Pollini'' -- Padova}
}
\date[05/03/2018]{Il Nastro Magnetico: ieri, oggi, domani -- DEI/G Padova 5 marzo 2018}

\begin{document}
\newcounter{ms}
  
\begin{frame}
	\titlepage
\end{frame}
  
%
% Who was Scelsi?
%
\section{\Scelsi}

\pgfdeclareimage[width=0.3\textwidth]{scelsi}{\imagedir/scelsi}
\begin{frame}
	\frametitle{Chi era \Scelsi?}

 	\begin{columns}[T]
      \begin{column}{0.3\textwidth}
          \begin{center}
              \pgfuseimage{scelsi}
 		      \end{center}
 	  \end{column}
    \begin{column}{0.65\textwidth}
    {
      \usebeamerfont{smalllist}
			\begin{itemize}%[<+- | alert@+->]
	
				\item Compositore italiano
				\item Nato a La Spezia nel 1905, morto a Roma nel 1988
	      \item Uno dei primi utilizzatori della dodecafonia in Italia
				\item Lunga crisi creativa negli anni '50
				\item Rivoluzione dei suoi metodi e delle sue prassi compositive
				\item Improvvisazioni registrate su vinile e poi su nastro
	            trascritte da ``copisti'' su commissione
	      \item Risultati musicali ``tra'' Var\`ese e Xenakis
	      \item Ossessivamente ostacolato e ignorato in Italia, riconosciuto nel
	          mondo intero dagli anni '70
	      \item Misteriose le traiettorie creative delle sue composizioni
	
			\end{itemize}
		}
		\end{column}
  \end{columns}

\end{frame}

\pgfdeclareimage[width=0.3\textwidth]{scelsi_giovane}{\imagedir/Scelsi_giovane}
\begin{frame}
	\frametitle[Metodologie Compositive]{Metodologie Compositive di \Scelsi}

 	\begin{columns}[T]
      \begin{column}{0.3\textwidth}
          \begin{center}
              \pgfuseimage{scelsi_giovane}
 		      \end{center}
 	  \end{column}
    \begin{column}{0.65\textwidth}
    \usebeamerfont{smalllist}
				\begin{itemize}%[<+- | alert@+->]
		
	        \item Sino agli inizi degli anni '50 scrittura tradizionale
	
	        \item Dagli inizi degli anni '50 in poi:
	
	            \begin{itemize}%[<+- | alert@+->]
                \usebeamerfont{smalllist}

	                \item  improvvisazioni al pianoforte o all'ondiola o altri
	                    strumenti
	                \item registrate prima su dischi di cera
	                \item poi su nastro magnetico
	                \item consegnate a ``copisti'' di fiducia per la
	                    trascrizione, \onslide<+- | alert@+->
	                    con indicazioni sugli strumenti da usare, segmenti da
	                    prelevare ecc.
	                \item rivedute e corrette assieme ai ``copisti'' stessi
	                \item edite da \emph{Salabert} su indicazione di Xenakis a
	                    partire dagli anni '70 in poi

	            \end{itemize}
				\end{itemize}
		\end{column}
  \end{columns}

\end{frame}

\begin{frame}
	\frametitle[Materiali]{I Materiali del ``fare compositivo di \Scelsi}

  \begin{itemize}%[<+- | alert@+->]

      \item Le registrazioni sono usate come ``cahiers d'esquisses'' 

      \item Esse servono a Scelsi per mediare quell'insopportabile distanza
          che c'\`e tra l'invenzione creativa e la sua realizzazione compiuta
          \onslide<+- | alert@+->(egli diceva di sentirsi solo un
          ``intermediario'' di questo flusso creativo)

      \item Scelsi non era n\'e un tecnico n\'e tantomeno un audiofilo\dots

      \item \dots anche se era molto attento al timbro e ai suoni

      \item Nelle trascrizioni, chiedeva che venissero trascritti \emph{tutti}
          gli eventi sonori \onslide<+- | alert@+->(non solo quelli
          creati da lui stesso, ma anche gli altri -- incidentali --
          registrati in quel momento)

  \end{itemize}

\end{frame}

\section{L'Archivio FIS}

\begin{frame}
	\frametitle{L'Archivio}

	\pause
	\begin{itemize}%[<+- | alert@+->]

		\item L'archivio della \FIS contiene

			\begin{itemize}%[<+- | alert@+->]

				\item documenti cartacei (partiture, articoli, lettere)

				\item 270 nastri magnetici di interesse primario

				\item una quantit\`a ancora imprecisata di nastri magnetici di
				      interesse secondario

			\end{itemize}

		\item Per un recupero ed una manutenzione appropriata, i nastri vanno
		      riversati su supporto digitale ad alta definizione
			  (campionamento 24--bit/96 kHz)

		\item Priorit\`a: nastri di interesse primario

	\end{itemize}

\end{frame}

\subsection{Condizioni Originali}
\begin{frame}
	\frametitle{Condizioni originali di registrazione}

	\begin{itemize}%[<+- | alert@+->]

		\item Registrazioni via microfono

		\item Sorgenti varie: pianoforte, ondiola, altri strumenti, radio,
		      giradischi per supporti in vinile, lacca, ecc.

		\item Scarsa informazione ancillare sulle velocit\`a di scorrimento,
		      apparati di registrazione, senso di scorrimento, ecc.

		\item Qualit\`a di registrazione molto scarsa (talvolta pessima)

	    \item Rumori dell'ambiente circostante (risonanza della stanza, rumori
			  ambientali -- uccellini, automobili, ecc.)

		\item Fonti sonore accidentali (spezzoni di emissioni radiofoniche
		      databili, ecc.)

	\end{itemize}

\end{frame}

\subsection{Peculiarit\`a}
\setcounter{ms}{1}
\begin{frame}
    \frametitle{Peculiarit\`a del recupero (\arabic{ms})}

	
	\begin{itemize}%[<+- | alert@+->]

		\item Difficolt\`a di identificazione di elementi chiave necessari al
		      riversamento stesso (velocit\`a e verso di scorrimento)

		\item Contesto creativo fortemente sperimentale (ricerca sulle forme,
		sull'articolazione, sulla tecnologia stessa)

		\item Utilizzazione dei \emph{suoni ancillari} (rumori legati alla
		tecnologia stessa, riverbero della stanza, passaggio delle automobili,
		canto degli uccelli, ecc.) per l'accumulo di informazioni importanti
		(velocit\`a e verso di scorrimento, stagione, ora del giorno, stanza,
		abitazione, data di registrazione, ecc.)

	\end{itemize}

\end{frame}

\refstepcounter{ms}
\begin{frame}
    \frametitle{Peculiarit\`a del recupero (\arabic{ms})}
	
	\begin{itemize}%[<+- | alert@+->]

		\item Preservazione dettagliata dello \emph{stato acustico} del
		      nastro (nessuna operazione di restauro)

    \item Trascrizioni multiple (ove necessario)

		\item Ipotesi di disposizione logica dei materiali su ciascun reperto

	\end{itemize}

\end{frame}

\subsection{Tecniche}
\begin{frame}
	\frametitle{Tecniche di identificazione}

	\begin{itemize}%[<+- | alert@+->]

    \item Misurazione del \emph{hum}

		\item \emph{Machine Fingerprinting}: registrazione dei rumori
		ancillari dei registratori di propriet\`a di \Scelsi

		\item Misurazione del riverbero caratteristico della stanza

		\item Correlazione dei rumori presenti sui nastri con quelli
		      registrati durante le sessioni di \emph{machine
			  fingerprinting}

		\item Ricerche in rete e in archivio di riscontri documentari
		      e/o acustici

	\end{itemize}

\end{frame}

\section{Esempi}

\setcounter{ms}{1}
\begin{frame}
	\frametitle{Esempio \Roman{ms}}

	\begin{center}
	\href{run:\exampledir/33bis-A04@9,5.aup}
  {\pgfimage[height=0.55\textheight]{\imagedir/33bis-A04@9,5}}
	\end{center}

	\begin{itemize}%[<alert@+->]

		\item Sorgente del suono?

		\item Casualit\`a o Determinazione?

	\end{itemize}

\end{frame}

\stepcounter{ms}
\begin{frame}
	\frametitle{Esempio \Roman{ms}}

	\begin{center}
	\href{run:\exampledir/33bis-A05@9,5.aup}
  {\pgfimage[height=0.55\textheight]{\imagedir/33bis-A05@9,5}}
	\end{center}

	\begin{itemize}%[<alert@+->]

		\item Sorgente del suono?

		\item Casualit\`a o Determinazione?

	\end{itemize}

\end{frame}

\stepcounter{ms}
\begin{frame}
	\frametitle{Esempio \Roman{ms}}

	\begin{center}
	\href{run:\exampledir/NMGS078-591-A10+A11-B06+B07.aup}
  {\pgfimage[height=0.55\textheight]{\imagedir/NMGS078-591-A10+A11-B06+B07}}
	\end{center}

	\begin{itemize}%[<alert@+->]

		\item Verso di scorrimento?

		\item Casualit\`a o Determinazione?

	\end{itemize}

\end{frame}

\stepcounter{ms}
\begin{frame}
	\frametitle{Esempio \Roman{ms}}

	\begin{center}
    \pgfimage[height=0.8\textheight]{\imagedir/591-002}
	\end{center}

	\begin{itemize}

		\item Scritte sui contenitori

	\end{itemize}

\end{frame}

\stepcounter{ms}
\begin{frame}
	\frametitle{Esempio \Roman{ms}}

	\begin{center}
    {\pgfimage[height=0.8\textheight]{\imagedir/Snapshot33bis}}
	\end{center}

	\begin{itemize}

		\item Layout grafico dell'``interpretazione'' logica del riversamento

	\end{itemize}

\end{frame}

% \section{L'Archiviazione}
% 
% \subsection{La Scheda}
% \begin{frame}
% 	\frametitle{Scheda Catalografica}
% 
% 	\begin{center}
% 	  \pgfimage[height=0.75\textheight]{\imagedir/db-structure}
% 	\end{center}
% 
% \end{frame}
% 
% \subsection{L'Archivio}
% \begin{frame}
% 	\frametitle{Struttura dell'Archivio}
% 
% 	\begin{itemize}[<+- | alert@+->]
% 	
% 		\item Struttura ibrida Gerarchica/Relazionale
% 
% 		\item Dati Multimediali
% 
% 		\item Predisposizione per lo standard MAG-XML
% 
% 	\end{itemize}
% 
% \end{frame}
% 
% \subsection{Il Sistema}
% 
% \setcounter{ms}{1}
% \begin{frame}
%     \frametitle{Sistema Informativo ({\arabic{ms}})}
% 
% 	\begin{center}
% 	  \pgfimage[height=0.75\textheight]{\imagedir/is}
% 	\end{center}
% 
% \end{frame}
% 
% \refstepcounter{ms}
% \begin{frame}
%     \frametitle{Sistema Informativo ({\arabic{ms}})}
% 
%     \begin{itemize}[<+- | alert@+->]
%         \item Server GNU/Linux
%         \item Piattaforma di sviluppo \emph{Ruby on Rails}
%         \item Database sql standard (\emph{mysql})
%         \item Software dedicato: {\tt fishrdb}
%         \item L'intero sistema \onslide<+- | alert@+-> software dedicato
%             incluso \onslide<+- | alert@+-> \`e sviluppato con Software Libero
%             ed \`e disponibile con licenza GNU/GPL 2
%         \item Sito di sviluppo: \hhref{http://trac.sme-ccppd.org/fishrdb}
%     \end{itemize}
% 
% \end{frame}

\setcounter{ms}{1}
\begin{frame}
    \frametitle{Il Lavoro di Trasferimento}

	\begin{itemize}%[<+- | alert@+->]
  \usebeamerfont{smalllist}

    \item Ad oggi, il lavoro di recupero digitale dell'Archivio della \FIS \`e
        stato interamente finanziato dalla Fondazione stessa

    \item Una prima fase \`e stata svolta da Nicola Bernardini e da Piero
        Schiavoni (Studio Coltempo, Roma)

    \item Dal Settembre 2007 in poi, la \FIS ha instaurato uno stretto
        rapporto di collaborazione con la Discoteca di Stato/Istituto
        Nazionale dell'Audiovisivo; Nicola Bernardini ha continuato ivi i
        trasferimenti dei nastri coadiuvato dai tecnici Bruno Quaresima e
        Carlo Cursi

    \item Nicola Bernardini ha creato e mantenuto sino al 2017 il sistema
        di archiviazione multimediale della \FIS ({\tt fishrdb})  coadiuvato dal Coordinatore
        dell'Archivio Mauro Tosti Croce e dalle archiviste Maria Natalina
        Trivisano, Marta Cardillo e Laura Piazza

    \item Nel 2017 l'archiviazione dei documenti cartacei \`e stata migrata al sistema
            adottato dal Ministero dei Beni Culturali
            (\emph{CollectiveAccess}); l'archiviazione dei documenti sonori
            \`e in via di migrazione

	\end{itemize}

\end{frame}

\section{Ringraziamenti}

\begin{frame}
    \frametitle{Desidero ringraziare\dots}

	  \begin{itemize}%[<+- | alert@+->]
    \usebeamerfont{smalllist}

	    \item la \FIS ed in particolare il suo Presidente Irmela Heimb\"acher Evangelisti,
	          la Coordinatrice della Ricerca Alessandra Carlotta Pellegrini e le
            archiviste Maria Natalina Trivisano, Laura Piazza e Marta Cardillo
	          per la pazienza mostrata nei miei confronti
	
	    \item Pietro Schiavoni (Studio Coltempo, Roma), co-autore
	        della strutturazione dei dati multimediali derivati dai trasferimenti
	
	    \item la Discoteca di Stato/Istituto
	        Nazionale dell'Audiovisivo ed in particolare il suo Direttore Massimo
	        Pistacchi e i tecnici Bruno Quaresima e Carlo Cursi per la
	        qualit\`a e la professionalit\`a della loro collaborazione
	
	    \item \emph{last but not least}, desidero ringraziare sentitamente lo specialista (e buon
	        amico) Prof.Sergio Canazza Targon per la pazienza con la quale ha
	        guidato i miei primi passi nel mondo del recupero dei suoni -- sempre
	        consigliandomi bene e portandomi sulla retta via
          (naturalmente mi assumo la responsabilit\`a
          di tutti gli errori)

	  \end{itemize}

\end{frame}

\end{document}
